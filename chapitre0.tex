\chapter{Applications linéaires et affines} 
Dans ce chapitre $E$ et $F$ désignent des $\K$-espaces vectoriels de dimension
quelconque.
\section{Applications linéaires} 
\subsection{Définitions} 
\begin{defdef} 
    Une application $f$ de $E$ dans $F$ est dite linéaire, ou bien que c'est un
    homomorphisme de $E$ dans $F$, si~: 
    \begin{equation} 
        \forall x,y \in E, \forall \lambda,\mu \in \K \quad f(\lambda x + \mu y) = \lambda f(x)+\mu f(y)
    \end{equation} 
\end{defdef} 
Cela équivaut aussi 
\begin{align}
    \forall x,y \in E \quad f(x+y) &= f(x)+f(y) \\
    \forall x \in E \forall \lambda \in \K \quad f(\lambda x)= \lambda f(x) 
\end{align}

On vérifie aisément que l'ensemble $\Lin{E}{F}$ des applications linéaires de
$E$ dans $F$ est un $\K$-espace vectoriel, sous-espace vectoriel de l'espace
vectoriel $E^F$ de toutes les applications de $E$ dans $F$.

\paragraph{Vocabulaire} 
Soit $f \in \Lin{E}{F}$. Si $f$, $E$ et $F$ sont quelconques alors c'est un
homomorphisme ; si $E=F$ et $f$ quelconque il s'agit d'un endomorphisme, leur
ensemble est $\Endo{E}$ ; si $f$ est bijective et $E$, $F$ quelconque alors $f$
est un isomorphisme, leur ensemble est $\Iso{E}{F}$ ; si $f$ est bijective avec
$E=F$ alors $f$ est un automorphisme, leur ensemble est $\GL{E}$.

\paragraph{Isomorphismes} 
On a deux résultats importants~: 
\begin{itemize} 
    \item Soit $f \in \Iso{E}{F}$. Deux cas sont possibles~: 
    \begin{itemize} 
        \item soit $E$ et $F$ sont de dimension infinie ; 
        \item soit $E$ et $F$ sont de dimension finie et $\dim{E} = \dim{F}$, on
            ne parle pas de dimension si on ne sait pas que l'espace est de
            dimension finie
    \end{itemize} 
    \item Deux espaces vectoriels de même dimension finie sont isomorphes : il
        existe un isomorphisme de l'un vers l'autre.
\end{itemize}

\subsection{Propriétés} 
\begin{prop} 
    Si $E$ et $F$ sont de dimension finie alors
    \begin{equation}
        \dim\Lin{E}{F}=\dim{E} \times \dim{F}
    \end{equation}
\end{prop}
%
\paragraph{Images directes et réciproques de sous-espaces vectoriels} 
\begin{prop}
    Soit \[f \in \Lin{E}{F}\] une application linéaire de $E$ dans $F$.
    \begin{enumerate} 
        \item Si $E'$ est un sous-espace vectoriel de $E$ alors $f(E') =
        \enstq{y \in F}{\exists x \in E' y=f(x)}$ est un sous-espace vectoriel
            de $F$ ;
        \item Si $F'$ est un sous-espace vectoriel de $F$ alors $f^{-1}(F') =
            \enstq{x \in E}{f(x) \in F'}$ est un sous-espace vectoriel de $E$.
    \end{enumerate}
\end{prop}
%
\begin{prop} 
    L'ensemble $\GL{E}$ des automorphismes de $E$ est un groupe pour la loi de
    composition $\circ$. On l'appelle le \emph{groupe linéaire} de l'espace vectoriel $E$.  
\end{prop}
%
\begin{prop} 
    L'ensemble $(\Endo{E},+,\circ;\dot)$ des endomorphismes de $E$ est une
    $\K$-algèbre. Elle n'est ni commutative ni intègre.
\end{prop}
\subsection{Image de familles} 
Soit $f \in \Lin{E}{F}$ et $\xi=(x_i)_{i \in I}$ une famille de vecteurs de $E$.
On note $f(xi) = (f(x_i))_{i \in I}$ la famille image de $\xi$ par $f$. On
vérifiera à titre d'exercice que~:
\begin{itemize}
    \item Si $\xi$ est liée alors $f(\xi)$ est liée ;
    \item Si $f(\xi)$ est libre alors $\xi$ est libre ;
    \item Si $\xi$ est génératrice dans $E$ alors $f(\xi)$ est génératrice dans
        $\Image{f}$ ;
    \item Si $\xi$ est génératrice dans $E$ et $f$ surjective alors $f(\xi)$ est
        génératrice dans $F$ ;
    \item Si $\xi$ est libre et $f$ injective alors $f(\xi)$ est libre ;
    \item Si $\xi$ est une base de $E$ et $f$ bijective alors $f(\xi)$ est une base de $F$.
\end{itemize}
%
\section{Applications affines}
\subsection{Définition}
\begin{defdef}
    Une application $f$ de $E$ dans $f$ est dite \emph{affine} s'il existe $a
    \in E$ et $\tilde{f} \in \Lin{E}{F}$ tels que~:
    \begin{equation}
        \forall x \in E \quad f(x)=f(a)+\tilde{f}(x-a)
    \end{equation}
    et avec des points $\forall M \in E \quad f(M)=f(A)+ \tilde{f}(\vect{AM})$.
\end{defdef}

On montre alors aisément que~:
\begin{itemize}
    \item $\tilde{f}$ est unique, on l'appelle la partie linéaire de $f$ ;
    \item quelque soient $x$ et $y$ de $E$, $f(x)=f(y)+\tilde{f}(x-y)$;
    \item $f$ est injective (resp. surjective, bijective) si et seulement si
        $\tilde{f}$ est injective (resp. surjective, bijective)
\end{itemize}

\paragraph{Exemples}
\begin{itemize}
    \item Une application linéaire est affine, sa partie linéaire est elle-même ;
    \item Une translation est affine, sa partie linéaire est l'identité de $E$
        (réciproque vraie).
\end{itemize}
\subsection{Propriétés}

\paragraph{Structure}
\begin{prop}
    L'ensemble $\A(E,F)$ des applications affines de $E$ dans $F$ est un
    sous-espace vectoriel de $\F(E,F)$.
\end{prop}

\paragraph{Image directes et réciproques de sous-espaces affines (sea)}
