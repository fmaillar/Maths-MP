\chapter{Applications linéaires et affines} 
Dans ce chapitre $E$ et $F$ désignent des $\K$-espaces vectoriels de dimension
quelconque.
\section{Applications linéaires} 
\subsection{Définitions} 
\begin{defdef} 
    Une application $f$ de $E$ dans $F$ est dite linéaire, ou bien que c'est un
    homomorphisme de $E$ dans $F$, si~: 
    \begin{equation} 
        \forall x,y \in E, \forall \lambda,\mu \in \K \quad f(\lambda x + \mu y) = \lambda f(x)+\mu f(y)
    \end{equation} 
\end{defdef} 
Cela équivaut aussi 
\begin{align}
    \forall x,y \in E \quad f(x+y) &= f(x)+f(y) \\
    \forall x \in E \forall \lambda \in \K \quad f(\lambda x)= \lambda f(x) 
\end{align}

On vérifie aisément que l'ensemble $\Lin{E}{F}$ des applications linéaires de
$E$ dans $F$ est un $\K$-espace vectoriel, sous-espace vectoriel de l'espace
vectoriel $E^F$ de toutes les applications de $E$ dans $F$.

\paragraph{Vocabulaire} 
Soit $f \in \Lin{E}{F}$. Si $f$, $E$ et $F$ sont quelconques alors c'est un
homomorphisme ; si $E=F$ et $f$ quelconque il s'agit d'un endomorphisme, leur
ensemble est $\Endo{E}$ ; si $f$ est bijective et $E$, $F$ quelconque alors $f$
est un isomorphisme, leur ensemble est $\Iso{E}{F}$ ; si $f$ est bijective avec
$E=F$ alors $f$ est un automorphisme, leur ensemble est $\GL{E}$.

\paragraph{Isomorphismes} 
On a deux résultats importants~: 
\begin{itemize} 
    \item Soit $f \in \Iso{E}{F}$. Deux cas sont possibles~: 
    \begin{itemize} 
        \item soit $E$ et $F$ sont de dimension infinie ; 
        \item soit $E$ et $F$ sont de dimension finie et $\Dim{E} = \Dim{F}$, on
            ne parle pas de dimension si on ne sait pas que l'espace est de
            dimension finie
    \end{itemize} 
    \item Deux espaces vectoriels de même dimension finie sont isomorphes : il
        existe un isomorphisme de l'un vers l'autre.
\end{itemize}

\subsection{Propriétés} 
\begin{prop} 
    Si $E$ et $F$ sont de dimension finie alors
    \begin{equation}
        \Dim\Lin{E}{F}=\Dim{E} \times \Dim{F}
    \end{equation}
\end{prop}
%
\paragraph{Images directes et réciproques de sous-espaces vectoriels} 
\begin{prop}
    Soit \[f \in \Lin{E}{F}\] une application linéaire de $E$ dans $F$.
    \begin{enumerate} 
        \item Si $E'$ est un sous-espace vectoriel de $E$ alors $f(E') =
        \enstq{y \in F}{\exists x \in E' y=f(x)}$ est un sous-espace vectoriel
            de $F$ ;
        \item Si $F'$ est un sous-espace vectoriel de $F$ alors $f^{-1}(F') =
            \enstq{x \in E}{f(x) \in F'}$ est un sous-espace vectoriel de $E$.
    \end{enumerate}
\end{prop}
%
\begin{prop} 
    L'ensemble $\GL{E}$ des automorphismes de $E$ est un groupe pour la loi de
    composition $\circ$. On l'appelle le \emph{groupe linéaire} de l'espace vectoriel $E$.  
\end{prop}
%
\begin{prop} 
    L'ensemble $(\Endo{E},+,\circ;\dot)$ des endomorphismes de $E$ est une
    $\K$-algèbre. Elle n'est ni commutative ni intègre.
\end{prop}
\subsection{Image de familles} 
Soit $f \in \Lin{E}{F}$ et $\xi=(x_i)_{i \in I}$ une famille de vecteurs de $E$.
On note $f(xi) = (f(x_i))_{i \in I}$ la famille image de $\xi$ par $f$. On
vérifiera à titre d'exercice que~:
\begin{itemize}
    \item Si $\xi$ est liée alors $f(\xi)$ est liée ;
    \item Si $f(\xi)$ est libre alors $\xi$ est libre ;
    \item Si $\xi$ est génératrice dans $E$ alors $f(\xi)$ est génératrice dans
        $\Image{f}$ ;
    \item Si $\xi$ est génératrice dans $E$ et $f$ surjective alors $f(\xi)$ est
        génératrice dans $F$ ;
    \item Si $\xi$ est libre et $f$ injective alors $f(\xi)$ est libre ;
    \item Si $\xi$ est une base de $E$ et $f$ bijective alors $f(\xi)$ est une base de $F$.
\end{itemize}
%
\section{Applications affines}
\subsection{Définition}
\begin{defdef}
    Une application $f$ de $E$ dans $f$ est dite \emph{affine} s'il existe $a
    \in E$ et $\tilde{f} \in \Lin{E}{F}$ tels que~:
    \begin{equation}
        \forall x \in E \quad f(x)=f(a)+\tilde{f}(x-a)
    \end{equation}
    et avec des points $\forall M \in E \quad f(M)=f(A)+ \tilde{f}(\vect{AM})$.
\end{defdef}

On montre alors aisément que~:
\begin{itemize}
    \item $\tilde{f}$ est unique, on l'appelle la partie linéaire de $f$ ;
    \item quelque soient $x$ et $y$ de $E$, $f(x)=f(y)+\tilde{f}(x-y)$;
    \item $f$ est injective (resp. surjective, bijective) si et seulement si
        $\tilde{f}$ est injective (resp. surjective, bijective)
\end{itemize}

\paragraph{Exemples}
\begin{itemize}
    \item Une application linéaire est affine, sa partie linéaire est elle-même ;
    \item Une translation est affine, sa partie linéaire est l'identité de $E$
        (réciproque vraie).
\end{itemize}
\subsection{Propriétés}

\paragraph{Structure}
\begin{prop}
    L'ensemble $\A(E,F)$ des applications affines de $E$ dans $F$ est un
    sous-espace vectoriel de $\F(E,F)$.
\end{prop}

\paragraph{Image directes et réciproques de sous-espaces affines (sea)}
\begin{prop}
	Soit $f \in \A(E, F)$ une application affine de E dans F.
	\begin{enumerate}
		\item Si $A$ est un sea de $E$ de direction E' alors $f(A) = \enstq{y \in F}{\exists x \in A \ y= f(x)}$ est un sea de $F$ de direction $f(E')$ ;
		\item Si $B$ est un sea de $F$ de direction $F'$ alors $J^{1}(B) = \enstq{x \in E}{f(x) \in B}$ est ou bien vide ou bien un sea de E de direction $J^{1}(F')$.
	\end{enumerate}
\end{prop}

\paragraph{Composition}
\begin{prop}
	Soient $E, F, G$ des $\K$-espaces vectoriels et $f \in \A(E, F), g \in \A(F, G)$, alors $g \circ f \in \A(E, G)$, avec~:
	\begin{equation}
		\widetilde{g \circ f} = \tilde{g} \circ \tilde{f}
	\end{equation}
\end{prop}

\paragraph{Conservation du barycentre}
\begin{prop}
	Soit $f \in \A(E, F)$ une application affine de E dans F. Si $g$ est le barycentre de la famille $(x_i)_{i \in I}$ affectée des coefficient $(\alpha_i)_{i \in I}$, alors $f(g)$ est le barycentre e la famille $(f(x_i))_{i \in I}$ affectée des coefficient $(\alpha_i)_{i \in I}$.
\end{prop}

\section{Image, noyau et rang}
Dans cette section, $E$, $F$ et $G$ sont des espaces vectoriels quelconques
\subsection{Rang d'une application linéaire}
\begin{defdef}
	Soit $f \in \Lin{E}{F}$ une application linéaire de $E$ dans $F$. L'espace vectoriel image de $f$ est~:
	\begin{equation}
		\Image{f} = \enstq{y \in F}{\exists x \in E \ y=f(x)}.
	\end{equation}
	Si $\Image{f}$ est de dimension finie, on dit que $f$ est de rang fini égal à la dimension de $f$~: $\rg{f} = \dim{\Image{f}}$ ; sinon on dit que $f$ est de rang infini.
\end{defdef}

\emph{Remarque}~: Il est clair que $f$ est surjective si et seulement si $\Image{f}=F$.

\begin{prop}
	Soit $f \in Lin{E}{F}$ et $(x_i)_{i \in I}$ une famille génératrice de $E$. Le rang de $f$ est aussi le rang de la famille de vecteurs $(f(x_i))_{i \in I}$
\end{prop}
\begin{proof}
	En effet, $\rg{f} = \Dim{\Image{f}} = \Dim{\VectEngendre{(f(x_i))_{i \in I}}} = \rg{(f(x_i))_{i \in I}}$
\end{proof}
\begin{prop}
	Soit $f \in Lin{E}{F}$. On a $\rg{f} \leq \minof{\Dim{E}}{\Dim{F}}$
\end{prop}
Lire la sous-section 17.4.2 \emph{Rang d’une application linéaire} du cours de MPSI pour les démonstrations des inégalités.

\begin{prop}
	Soient $f \in \Lin{E}{F}$ et $g \in \Lin{F}{G}$, alors~:
	\begin{enumerate}
		\item $\rg{g \circ f} \leq minof{\rg{f}}{\rg{g}}$ ;
		\item Si $f \in \Iso{E}{F}$, alors $\rg{g \circ f} = \rg{g}$ ;
		\item Si $f \in \Iso{F}{G}$, alors $\rg{g \circ f} = \rg{f}$.
	\end{enumerate}
\end{prop}
Lire la démonstration du théorème 17.22 du cours de MPSI pour la démonstration.

\subsection{Noyau d'une application linéaire}
\begin{defdef}
	Soit $f \in \Lin{E}{F}$ une application linéaire de $E$ dans $F$.  Le noyau de $f$ est le sous-espace vectoriel de $E$~:
	\begin{equation}
		\ker{f} = \enstq{x \in E}{f(x)=0_F} = f^{-1}(0_F).
	\end{equation}
\end{defdef}

On prouve facilement que $f$ est injective si et seulement si son noyau est l'espace nul $\ker{f} = \{0_E\}$. Attention, cette équivalence n'est vraie que pour une application linéaire.

Si $E'$ est un sev de $E$ et si $f_{E'}$ est une restriction de $f$ à $E'$, alors 
\begin{equation}
	\ker{f_{E'}} = E' \cap \ker{f}
\end{equation}

\paragraph{Équation aux antécédents}
\begin{prop}
	Soit $f \in \A{E}{F}$, de partie linéaire $\tilde{f}$, et $b \in F$. L'ensemble $f^{-1}(b)$ des antécédents de $b$ par $f$ est soit vide soit un sous-espace affine de direction $\ker{f}$.
\end{prop}
\begin{proof}
	$f^{-1}(b)$ est vide si $b \not\in \Image{f}$, ce qui n'arrive pas lorsque $f$ est surjective. Lorsque $b \in \Image{f}$, alors il existe $a \in E$ tel que $b=f(a)$ et quelque soit $x \in E$ on a la suite d'équivalence~:
	\begin{equation}
		x \in f^{-1}(b) \iff f(x)=f(a) \iff \tilde{f}(x-a)=0_E \iff x-a \in \ker{\tilde{f}} \iff x \in a + \ker{\tilde{f}},
	\end{equation}
	ce qui permet d'écrire que $f^{-1}(b) = a + \ker{\tilde{f}}$ et donc que $f^{-1}(b)$ est bien un sous-espace affine de direction $\ker{f}$
\end{proof}

\paragraph{Théorème d'isomorphisme}
\begin{theo}
	Soit $f \in \Lin{E}{F}$ et $E'$ un supplémentaire de $\ker{f}$ dans $E$. L'application restreinte au départ à $E'$ et $\Image{f}$ à l'arrivée~:
	\begin{equation}
		g=f_{|E'}^{|\Image{f}} = \fonction{\tilde{f}}{E'}{\Image{f}}{x}{f(x)}
	\end{equation}
	est un isomorphisme d'espaces vectoriels.
\end{theo}
Lire la démonstration du théorème 16.13 du cours de MPSI, qui énonce que quelque soit $f \in \Lin{E}{F}$, tout supplémentaire du noyau de $f$ est isomorphe à l'image de $f$.
En conséquence, on a la formule du rang~:
\begin{corth}
	Si $E$ est de dimension finie et $F$ de dimension quelconque, pour toute application linéaire $f \in \Lin{E}{F}$~:
	\begin{equation}
		\Dim{E} = \Dim{\ker{f}} + \rg{f}.
	\end{equation}
\end{corth}
Lire la démonstration à la sous-section 17.4.1 du cours de MPSI.

La formule du rang donne immédiatement~:
\begin{corth}
	Si $E$ et $F$ sont de même dimension finie, on a pour toute $f \in \Lin{E}{F}$~: $f$ est bijective ssi $f$ est injective ssi $f$ est surjective.
\end{corth}
Lire la démonstration du théorème 17.23 du cours de MPSI.

\section{Caractérisation d'une application linéaire}

Les trois résultats suivants sont fondamentaux : ils donnent trois façons d'introduire une application linéaire. Il y en aura une quatrième à l'aide des matrices. Ce sont de bons exercices d'entraînement aux questions d'existence et d'unicité.

\begin{theo}
	Soient $E, F_1, \ldots, F_p$ des espaces vectoriels et $\fonction{f}{E}{F_1 \times \ldots \times F_p}{x}{(f_1(x), \ldots, f_p(x))}$ une application de $E$ dans l'espace vectoriel produit $F_1 \times \ldots \times F_p$. On a l'équivalence suivante~:
	\begin{equation}
		f \in \Lin{E}{F} \iff \forall i \in \intervalleentier{1}{p} \ f_i \in \Lin{E}{F_i}
	\end{equation}
Soit encore $f$ est linéaire si et seulement si $f_1, \ldots, f_p$ sont linéaires.
\end{theo}