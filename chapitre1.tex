\chapter{Matrices}
\label{chap:matrices}
\minitoc
\minilof
\minilot

\section{Présentation}
\label{chap1-sec:presentation}
\subsection{Définitions}
Une matrice de format $n \times p$ et à coefficients dans le corps $\K$ est une famille $A = (a_{i,j})_{1\leq i \leq n, 1 \leq j \leq p}$ d'éléments de $\K$ indexée par $\intervalleentier{1}{n} \times \intervalleentier{1}{p}$. On représente cette famille sous la forme d'un tableau~:
\begin{equation}
	A = \begin{pmatrix}
		a_{1,1} & a_{1,2} & \cdots & a_{1,p} \\
		a_{2,1} & a_{2,2} & \cdots & a_{2, p} \\
		\vdots & \vdots &  & \vdots \\
		a_{n,1} & a_{n,2} & \cdots & a_{n,p}
	\end{pmatrix}
\end{equation}
dans lequel $a_{i,j}$ est le coefficient se trouvant à l'intersection de la $i$\ieme{} ligne et de la $j$\ieme{} colonne, on dit que c'est le coefficient de la place $(i, j)$ dans $A$.

\begin{remarque}
	Il est fondamental dans les notations de bien distinguer~:
	\begin{itemize}
		\item $a_{i,j}$ qui est un élément de $\K$ pour un certain couple $(i, j)$
		\item $(a_{i,j})_{1\leq i \leq n, 1 \leq j \leq p}$ qui est un élément de $\Mnp{n}{p}{\K}$ et où $(i, j)$ qui est muet, n'a pas à être introduit \dots{}
	\end{itemize}
\end{remarque}
D'ailleurs, si $A = (a_{i,j})_{1\leq i \leq n, 1 \leq j \leq p}$, on a aussi bien $A = (a_{k,k})_{1\leq k \leq n, 1 \leq l \leq p}$, mais pas $A = (a_{i,j})_{1\leq j \leq n, 1 \leq i \leq p}$ car dans cette dernière matrice le coefficient de place $(1,2)$ par exemple est $a_{2,1}$ !

Formats particulier~:
\begin{itemize}
	\item Si $n=1$, il s'agit des matrices lignes $L = (a_1, \cdots, a_p)$ ;
	\item Si $p=1$, il s'agit des matrices colonnes $C = \begin{pmatrix} a_1 \\ \vdots \\ a_n \end{pmatrix}$ ;
	\item Si $n=p=1$, il s'agitdes scalaires $A=a_{1,1}=a$.
\end{itemize}

On note $\Mnp{n}{p}{\K}$ l'ensemble des matrices de format $n \times p$ et à coefficients dans le corps $\K$. Si $n=p$, on note simplement $\Mn{n}{\K}$ l'ensemble des matrices carrées de taille $n$ à coefficients dans $\K$.

\subsection{Opérations}
\paragraph{Somme et produit par un scalaire}
$\Mnp{n}{p}{\K}$ est naturellement muni d'une structure de $\K$-espace vectoriel, au titre d'ensemble des familles d'éléments de $\K$ indexées par $I = \intervalleentier{1}{n} \times \intervalleentier{1}{p}$. Les lois en sont~: Quelque soient les matrices $a$ et $b$ dans $\Mnp{n}{p}{\K}$ et un scalaire $\lambda$~:
\begin{align}
	(a_{i,j})_{1\leq i \leq n, 1 \leq j \leq p} + (b_{i,j})_{1\leq i \leq n, 1 \leq j \leq p} & = (a_{i,j}+b_{i,j})_{1\leq i \leq n, 1 \leq j \leq p} \\
	\lambda  (a_{i,j})_{1\leq i \leq n, 1 \leq j \leq p} &= (\lambda a_{i,j})_{1\leq i \leq n, 1 \leq j \leq p}
\end{align}
Le vecteur nul de $\Mnp{n}{p}{\K}$ est la matrice $0 = (0)_{1\leq i \leq n, 1 \leq j \leq p}$ dont tous les coefficients sont nuls. La base la plus simple, appelée \emph{base canonique}, est $C = (E_{i,j})_{1\leq i \leq n, 1 \leq j \leq p}$ où pour tout $(i,j) \in \intervalleentier{1}{n} \times \intervalleentier{1}{p}$, $E_{i,j}$ est la matrice de format $n \times p$ dont tous les coefficients sont nuls sauf celui de place $(i, j)$ qui vaut 1. On peut donc écrire~:
\begin{equation}
	\forall (i,j) \in \intervalleentier{1}{n} \times \intervalleentier{1}{p} \qquad E_{i,j} = (\delta_{k,i} \delta_{l,j})_{1\leq k \leq n, 1 \leq l \leq p}
\end{equation}
et si $A = (a_{i,j})_{1\leq i \leq n, 1 \leq j \leq p}$ alors $A = \sum_{i=1}^n \sum_{j=1}^p a_{i,j} E_{i,j}$. en conséquence on a la dimension de $\Mnp{n}{p}{\K}$~: $\Dim{\Mnp{n}{p}{\K}} = n \times p$.
\begin{remarque}
	La notation $E_{i,j}$ est ambigüe car elle n'indique pas le format de cette matrice et par exemple il y a autant de matrices $E_{1,2}$ que de formats de matrice (autant dire une infinité !). On devrait noter par exemple $E_{i,j}^{n\times p}$ pour qu'il n'y ait pas d'ambigüité mais on ne le fait pas quand le contexte est clair.
\end{remarque}
\paragraph{Produit de matrices}
On définit, moins naturellement, le produit d'une matrice $n \times p$ par une matrice $p \times q$ de la façon suivante. Si $A  = (a_{i,j})_{1\leq i \leq n, 1 \leq j \leq p} \in \Mnp{n}{p}{\K}$ et $B  = (b_{i,j})_{1\leq i \leq p, 1 \leq j \leq q} \in \Mnp{p}{q}{\K}$, alors on appelle le produit de $A$ par $B$ noté $AB$, la matrice $AB=(c_{i,j})_{1\leq i \leq n, 1\leq j \leq q} \in \Mnp{n}{q}{\K}$ de format $n \times q$ dont le coefficient général est défini par la formule~:
\begin{equation}
	\forall (i,j) \in \intervalleentier{1}{n} \times \intervalleentier{1}{q} \quad c_{i,j} = \sum_{k=1}^p a_{i,k}b_{k,j}.
\end{equation}

Il faut remarquer que $A$, $B$ et $AB$ sont trois matrices de formats différents en général et que le produit $AB$ n'est possible que si le nombre de colonnes de $A$ correspond au nombre de lignes de $B$.

On dit qu'on fait un produit ligne à colonne puisque $c_{i,j}$ est le produit de la $i$\ieme{} ligne de $A$ par la $j$\ieme{} colonne de $B$, d'ailleurs on a l'égalité entre matrices $1 \times 1$~:
\begin{equation}
	\forall (i,j) \in \intervalleentier{1}{n} \times \intervalleentier{1}{q} \quad (c_{i,j}) = (a_{i,1}, \cdots, a_{i,k}, \cdots, a_{i,p}) \begin{pmatrix} b_{1,j} \\ \vdots \\b_{k,j} \\ \vdots \\b_{p,j} \end{pmatrix}
\end{equation}

Produit des vecteurs de la base canonique de $\Mn{n}{\K}$~: Les matrices $E_{i,j}$ étant carrées de taille $n$ on a~:
\begin{equation}
	\forall i,j,k,l \in \intervalleentier{1}{n} \quad E_{i,j}E_{k,l} = \delta_{j,k} E_{i,l}
\end{equation}

Pour les trois lois présentées ci-dessus $(\Mn{n}{\K},+,\cdot;\times)$ est une $\K$-algèbre. Son élément unité est la matrice identité~:
\begin{equation}
	I_n = \begin{pmatrix}
		1 & 0 & \cdots & 0 \\
		0 & 1 & \ddots & \vdots \\
		\vdots & \ddots & \ddots & 0 \\
		0 & \cdots & 0 & 1
	\end{pmatrix}
\end{equation}