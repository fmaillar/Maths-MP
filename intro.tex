\chapter*{Introduction}

L'adage dit : "l'avenir appartient à ceux qui se lèvent tôt" et c'est la vérité que ceux qui passent devant au
concours sont (à compétances égales) ceux (et celles) qui travaillent quand d'autres ne le font pas !
L'année de MathSpé est très courte : 25 semaines ! Réussissent aux concours les étudiants qui savent se motiver
très tôt. Il n'est pas pensable, pour qui veut intégrer une école d'ingénieur, de passer l'été sans travailler un
peu, beaucoup, passionnément. Pour bien démarrer l'année , j'attends de vous ce qui suit :

\section*{Un travail général : Un entretien indispensable de votre cours de Math-Sup.}
\begin{itemize}
\item Apprenez les définitions et divers théorèmes.
\item Relisez les démonstrations : ne les suivez pas mot à mot, ne vous emprisonnez pas dans du "par coeur"
mais comprenez l'idée qui les porte et exercez-vous à y apporter des variantes de notation , de choix d'objets
utilisés, des exemples ou contrexemples simples ...
Sans doute n'allez-vous pas travailler toutes les démonstrotions mais entraînez-vous par-ci par-là : elles sont
truffées de bonnes idées , d'exemples et de logique!
\item Faites des exercices très simples d'application du cours et assurez-vous des techniques de calcul :
nombres complexes, polynômes, intégration, développement limités, équations différentielles, tracé d'arcs, \ldots
Pour vous y inciter vous trouverez ci-joint une liste de 34 exercices (parI13i tant d'autres!) à rédiger sur
copies doubles, pour vous, proprement, travail qu'il faudra me montrer le jour de la rentrée . Il ne sera pas
ramassé mais me donnera ma première opinion sur vous, sur votre investissement dans voe études et votre
volonté d'intégrer une école d'ingénieur. Nous pourrons éventuellement (à la demande) corriger les plus
"coriaces" d'entre eux.
\item Sachez dès maintenant que durant l'année j'attacherai la plus grande importance à la présentation des
copies et à vos notes de cours. Je ne m'encombrerai pas de copies mal présentées ou illisibles ou encore truffées
de fautes d'orthographe. Une note forfaitaire de Osera attribuée dans ce cas.
\emph{Je ne m'intéresserai pas à ceux ou celles qui ne s'intéressent pas à ce qu'ils font !}
\end{itemize}
\section*{Un travail particulier : Un échauffement salutaire en algèbre linéaire.}
L'algèbre linéaire étant l'une des parties difficiles à assimiler en mathématiques, il est nécessaire de s'y prendre
à plusieurs reprises et de profiter de l'été pour revoir, au calme, ces notions fondamentales, avec un peu
d'autonomie et de distance. Je vous demande donc de préparer la première quinzaine de travail en classe
MP en travaillant à fond le cours et les exercices d'algèbre linéaire de Madame RAINERO. Assurez-vous
notamment que vous avez compris intimement le lien application linéaire-matrice et faites les 30 exercices
ci-joints (dont beaucoup sont três élémentaires).
Pour vous motiver dans ces révisions sachez que le jour de la rentrée ou le lendemain, après les présentations
d'usage, vous aurez un contrôle (d'une heure ou deux) portant sur des questions de cours d'algèbre linéaire et
sur les exercices distribués.
Je vous conseille par ailleurs, bien qu'il y en ait au CDI, l'achat de livres d'exercices corrigés pour vous
entraîner et préparer vos colles.

Vous êtes dans la période charnière de votre vie. Comprenez donc que c'est maintenant ou jamais que vous
pouvez vous donner la chance d'intégrer une école d'ingénieur qui vous intéresse. Si vous ne faites pas l'effort
nécessaire vous serez orienté par défaut dans une école qui ne vous plait pas ou encore à l'université ou en IUT.
De plus, les élèves qui n'auront pas travaillé ne seront pas autorisés à redoubler.

En deux mois il y a heureusement aussi largement le temps de passer de bons moments de détente alors : BONNES VACANCES!

N'oubliez pas non plus de lire les oeuvres de français au programme et des journaux en anglais ou en allemand ...

Alain Walbron.
