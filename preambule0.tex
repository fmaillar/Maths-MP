%pour résoudre le problème "No room for a new dimen" et on le place en premier
\usepackage{etex}
%
%packages fondamentaux
\usepackage[utf8]{inputenc}
\usepackage[T1]{fontenc}
\usepackage[francais]{babel}
%
%pour les symboles mathématiques
\usepackage{amsmath}
\usepackage{amsthm}
\usepackage{amssymb}
\usepackage{amsfonts}
\usepackage{mathrsfs}
\usepackage{latexsym}
%
%Pour faire des maths en francais
%\usepackage{frmath}
%
%Pour utiliser les tableaux
\usepackage{array}
%
%fonte vectorielle
\usepackage{lmodern}
%
%
%Pour écrire de la physique avec les unités
%\usepackage{siunitx}
%
%Pour inclure des images avec \includegraphics[scale=]{}
\usepackage{graphics}
%
%Pour inclure des fichiers .ps ou .eps
\usepackage{pstricks-add}
%
%Pour dessiner avec Tikz
\usepackage{tikz, pgf}
\usetikzlibrary{arrows,calc,backgrounds}
%
% pour le symbole \danger
\usepackage{fourier-orns}
%
%pour utiliser les couleurs
\usepackage{color}
%
%pour utiliser "draft" (brouillon)
%\usepackage{draftcopy}
%
%pour renvoyer les \es à la fin
\usepackage{endnotes}
%
%pour les encadrements et les boites
\usepackage{fancybox}
%
%pour les boucles de programmation en \LaTeX
\usepackage{ifthen}
%
%Pour le placement de flottants avec FloatBarrier
\usepackage{placeins}
%
%Pour changer les titres de section
%\makeatletter
%	\renewcommand\section{\@startsection
%	{section}{2}{0mm}
%	{-\baselineskip}{0.5\baselineskip}
%	{\FloatBarrier\normalfont\Large\bfseries}}
%\makeatother
%
%
%Calculs arithmétique
%\usepackage{calc}
%
%Lettrines
\usepackage{lettrine}
%pour le symbole double crochet \rrbracket \llbracket
\usepackage{stmaryrd}
%pour les minitoc{}
\usepackage{minitoc}
\mtcselectlanguage{french}
%
%pour les tableaux de variations
\usepackage{variations}
%
%Pour la taille de la page
\usepackage{geometry}
%\geometry{
%	a4paper,
%%	total={170mm,257mm},
%%	left=20mm,
%%	top=20mm,
%%	margin=2cm,
%}
%% entêtes et pieds de pages
%Entêtes et pieds de pages
\usepackage{fancyhdr}
\pagestyle{fancy}
%
\fancyhead{}
\fancyfoot{}
%
\renewcommand{\subsectionmark}[1]{%
  \ifsubsectioninheader
    \def\subsectiontitle{: #1}%
  \else
    \def\subsectiontitle{}%
  \fi}
\newif\ifsubsectioninheader
\def\subsectiontitle{}
\fancyhead[RO]{\nouppercase{\rightmark\ifsubsectioninheader\subsectiontitle\fi}}
%\fancyhead[RE]{MPSI F. M\bsc{aillard}}
\fancyhead[LE]{\nouppercase{\rightmark\ifsubsectioninheader\subsectiontitle\fi}}
%\fancyhead[LO]{MPSI F. M\bsc{aillard}}
\fancyfoot[LE]{\thepage}
\fancyfoot[RO]{\thepage}
%
%\setlength{\headheight}{15.36pt}
\setlength{\headheight}{13.6pt}
%
%pour inclure des pdf
\usepackage[final]{pdfpages}
%
%Pour l'interligne
%\linespread{1.1}
%
%texte en latin
%\usepackage{lipsum}
%
%pour creer un sommaire
\usepackage{shorttoc}
%
%pour écrire des arc géométriques
\usepackage{yhmath}
%
%pour l'index
%\usepackage{makeidx}
%\makeindex
%
%paquet pour les liens hypertextes
\usepackage[colorlinks=true,%
linkcolor=black,%
urlcolor=black,%
citecolor=black,%
pdftitle={Mathématiques},%
pdfauthor={FlorianMaillard},%
breaklinks = true%
]{hyperref}
%pour les tableaux
\usepackage{tabularx}
%onchange le nom de la table
\renewcommand{\tablename}{\bsc{Tableau}}
%TODO
\usepackage{todonotes}
%\usepackage{ulem}
%%%%%%%%%%%%%%%%%%%%%%% THM et STYLES %%%%%%%%%%%%%%%%%%%%%%
%
\newtheorem*{remarque}{Remarque}
\theoremstyle{plain} \newtheorem{theo}{Théorème}[chapter]
%\newtheorem*{theo_etoile}{Théorème}
\newtheorem{lemme}{Lemme}[chapter]
\theoremstyle{definition} \newtheorem{defdef}{Définition}[chapter]
%\newtheorem*{defdef_etoile}{Définition}
\newtheorem{prop}{Proposition}[chapter]
\newtheorem{rappel}{Rappel}[chapter]
\newtheorem{proprietes}{Propriétés}[chapter]
\theoremstyle{plain} \newtheorem{cor}{Corollaire}[prop]
\newtheorem{corth}{Corollaire}[theo]
%
%%%%%%%%%%%%%%%%%%%%%%%% COMMANDES PERSOS %%%%%%%%%%%%%%%
%
\newcommand{\fonction}[5]{#1\colon \left\{\begin{array}{ccc}
#2 & \longrightarrow &#3\\
#4 & \longmapsto & #5
\end{array}
\right.
}
%nouveau raccourcis
\newcommand{\ensemblenombre}[1]{\mathbb{#1}}
\newcommand{\intervalle}[4]{\mathopen{#1}#2\mathclose{}\mathpunct{};#3\mathclose{#4}}
\newcommand{\pv}{\ensuremath{\, ; }}
\newcommand{\intervalleff}[2]{\ensuremath{\left[ #1 \pv #2 \right]}}%{\intervalle{[}{#1}{#2}{]}}
\newcommand{\intervalleof}[2]{\ensuremath{\left] #1 \pv #2 \right]}}%{\intervalle{]}{#1}{#2}{]}}
\newcommand{\intervallefo}[2]{\ensuremath{\left[ #1 \pv #2 \right[}}%{\intervalle{[}{#1}{#2}{[}}
\newcommand{\intervalleoo}[2]{\ensuremath{\left] #1 \pv #2 \right[}}%{\intervalle{]}{#1}{#2}{[}}
\newcommand{\intervalleentier}[2]{\intervalle\llbracket{#1}{#2}\rrbracket}
\newcommand{\Rplus}{\intervallefo{0}{+\infty}}
\newcommand{\Rmoins}{\intervalleof{-\infty}{0}}
\newcommand{\Rplusetoile}{\intervalleoo{0}{+\infty}}
\newcommand{\Rmoinsetoile}{\intervalleoo{-\infty}{0}}
\newcommand{\N}{\ensemblenombre{N}}
\newcommand{\R}{\ensemblenombre{R}}
\newcommand{\Rbar}{\bar{\R}}
\newcommand{\K}{\ensemblenombre{K}}
\renewcommand{\L}{\ensemblenombre{L}}
\newcommand{\Z}{\ensemblenombre{Z}}
\newcommand{\ZZ}{\mathcal{Z}}
\newcommand{\C}{\ensemblenombre{C}}
\newcommand{\Q}{\ensemblenombre{Q}}
\newcommand{\A}{\mathcal{A}}
\renewcommand{\S}{\mathcal{S}}
\renewcommand{\P}{\mathcal{P}}
\newcommand{\I}{\mathcal{I}}
\newcommand{\Rep}{\mathcal{R}}
\newcommand{\Dr}{\mathcal{D}}
\newcommand{\Def}[1]{\Dr_{#1}}
\newcommand{\U}{\ensemblenombre{U}}
\newcommand{\Part}{\mathfrak{P}}
\renewcommand{\H}{\mathcal{H}}
\newcommand{\E}{\mathcal{E}}
\newcommand{\F}{\mathcal{F}}
\newcommand{\Gb}{\mathcal{G}}
\newcommand{\con}[1]{\mathcal{C}_{#1}}
\newcommand{\courbe}{\mathcal{C}}
%\newcommand{\cr}{\mathcal{C}}
\newcommand{\vect}[1]{\overrightarrow{#1}}
\newcommand{\vi}{\vect{i}}
\newcommand{\vj}{\vect{j}}
\newcommand{\vk}{\vect{k}}
\newcommand{\vu}{\vect{u}}
\newcommand{\vv}{\vect{v}}
\newcommand{\vw}{\vect{w}}
\newcommand{\vx}{\vect{x}}
\newcommand{\vy}{\vect{y}}
\newcommand{\vz}{\vect{z}}
\newcommand{\vn}{\vect{n}}
\newcommand{\va}{\vect{a}}
\newcommand{\vb}{\vect{b}}
\newcommand{\vc}{\vect{c}}
\newcommand{\rond}{(O,\vi,\vj)}
\newcommand{\bond}{(\vi,\vj)}
\newcommand{\rondtrois}{(O,\vi,\vj,\vk)}
\newcommand{\bondtrois}{(\vi,\vj,\vk)}
\newcommand{\rondtroisu}{(\Omega,\vu,\vv,\vw)}
\newcommand{\Ell}{\mathcal{E}}
%\renewcommand{\Re}{\mathfrak{R}}
%\renewcommand{\Im}{\mathfrak{I}}
\newcommand{\Lin}[2]{\mathcal{L}\left(#1,#2\right)}
\newcommand{\Aff}[2]{\mathcal{A}\left(#1,#2\right)}
\newcommand{\pLin}[3]{\mathcal{L}_{#1}\left(#2,#3\right)}
\newcommand{\Endo}[1]{\mathcal{L}\left(#1\right)}
\newcommand{\pEndo}[2]{\mathcal{L}_{#1}\left(#2\right)}
\newcommand{\Pass}[2]{\text{Pass}\left(#1,#2\right)}
\newcommand{\Isom}[2]{\mathbf{Isom}\left(#1,#2\right)}
\newcommand{\Iso}[2]{\mathbf{Iso}\left(#1,#2\right)}
\newcommand{\Is}[1]{\mathbf{Is}\left(#1\right)}
\newcommand{\Isplus}[1]{\mathbf{Is}^{+}\left(#1\right)}
\newcommand{\Ismoins}[1]{\mathbf{Is}^{-}\left(#1\right)}
\newcommand{\Auto}[1]{\mathbf{Aut}\left(#1\right)}
\newcommand{\Orth}[1]{\mathcal{O}\!\left(#1\right)}
\newcommand{\SOrth}[1]{\mathcal{SO}\!\left(#1\right)}
\newcommand{\Orthmin}[1]{\mathcal{O}^{-}\!\left(#1\right)}
\newcommand{\GL}[1]{\mathcal{GL}\left(#1\right)}
\newcommand{\Ga}[1]{\mathbf{Ga}\left(#1\right)}
\newcommand{\GLn}[2]{\mathcal{GL}_{#1}\left(#2\right)}
\newcommand{\uinv}[1]{\mathfrak{U}\left(#1\right)}
\newcommand{\RelBin}{\mathfrak{R}}
\newcommand{\X}{\mathcal{X}}
\newcommand{\Y}{\mathcal{Y}}
\newcommand{\CM}{\mathcal{CM}}
\newcommand{\Born}{\mathcal{B}}
\newcommand{\Mat}{\mathrm{Mat}}
\newcommand{\Mnp}[3]{\mathcal{M}_{#1,#2}\left(#3\right)}
\newcommand{\Mn}[2]{\mathcal{M}_{#1}\left(#2\right)}
\newcommand{\M}[2]{\mathcal{M}\left(#1;#2\right)}
\newcommand{\SN}[2]{\mathcal{S}_{#1}\left(#2\right)}
\newcommand{\AN}[2]{\mathcal{A}_{#1}\left(#2\right)}
\newcommand{\B}{\mathcal{B}}
\newcommand{\Lb}{\mathcal{L}}
\newcommand{\G}{\mathcal{G}}
\newcommand{\diff}{\mathop{}\mathopen{}\D}
\newcommand{\abs}[1]{\left\lvert#1\right\rvert}
\newcommand{\norme}[1]{\left\lVert#1\right\rVert}
\newcommand{\enstq}[2]{\left\{#1\mathrel{}\middle|\mathrel{}#2\right\}}
\newcommand{\petito}[1]{o\mathopen{}\left(#1\right)}
\newcommand{\grandO}[1]{O\mathopen{}\left(#1\right)}
\newcommand{\petitof}[2]{o\mathopen{}_{#1}\left(#2\right)}
\newcommand{\grandOf}[2]{O\mathopen{}_{#1}\left(#2\right)}
\newcommand{\contpoint}[3]{\mathcal{C}_{#1}(#2,#3)}
\newcommand{\cont}[2]{\mathcal{C}(#1,#2)}
\newcommand{\Lip}[2]{\mathcal{L}ip(#1,#2)}
\newcommand{\derivpt}[3]{\mathcal{D}_{#1}(#2,#3)}
\newcommand{\deriv}[2]{\mathcal{D}(#1,#2)}
\newcommand{\classe}[1]{\mathcal{C}^{#1}}
\newcommand{\derivnfois}[3]{\mathcal{D}^{#1}(#2,#3)}
\newcommand{\prodscal}[2]{\left\langle{#1},{#2}\right\rangle}
\newcommand{\Sim}[1]{\mathbf{Sim}\left(#1\right)}
\newcommand{\Simplus}[1]{\mathbf{Sim}^{+}\left(#1\right)}
\newcommand{\Simmoins}[1]{\mathbf{Sim}^{-1}\left(#1\right)}
\newcommand{\Bbar}[3]{\bar{B}_{#3}\left(#1, #2\right)}
\newcommand{\Vvois}{\mathcal{V}}
\newcommand{\Oouvert}{\mathfrak{O}}
\newcommand{\Ddomaine}{\mathfrak{D}}
\newcommand{\Transpose}[1]{#1^{\mathsf{T}}}

%nouveaux opérateurs
\DeclareMathOperator{\sgn}{signe}% fonction signe
\DeclareMathOperator{\e}{e}% exponentielle
\DeclareMathOperator{\argch}{argcosh}% argument cosinus hyperbolique
\DeclareMathOperator{\argsh}{argsinh}% argument sinus hyperbolique
\DeclareMathOperator{\argth}{argtanh}% argument tangente hyperbolique
\DeclareMathOperator{\cotan}{cotan}% cotangente
\DeclareMathOperator{\sinc}{sinc}% sinus cardinal
\DeclareMathOperator{\Arg}{Arg}% argument principal
\DeclareMathOperator{\ii}{i}% unité imaginaire
\DeclareMathOperator{\Ent}{E}% fonction partie entière
\DeclareMathOperator{\Id}{Id}% fonction identité
\DeclareMathOperator{\Det}{Det}% déterminant
\DeclareMathOperator{\Inf}{inf}% le plus grand des minorants
\DeclareMathOperator{\Card}{Card}% le cardinal d'un ensemble
\DeclareMathOperator{\Ker}{Ker}% le noyau d'une application linéaire
\DeclareMathOperator{\Image}{Im}% l'image d'une application linéaire
\DeclareMathOperator{\Div}{Div}% l'ensemble des diviseurs d'un entier
\DeclareMathOperator{\divgc}{div}% divergence d'un champs de vecteurs
\DeclareMathOperator{\cotanh}{coth}% cotangente hyperbolique
\DeclareMathOperator{\htan}{th}% tangente hyp
\DeclareMathOperator{\hcotan}{coth}% cotangente hyp
\DeclareMathOperator{\hsin}{sh}% sin hyp
\DeclareMathOperator{\hcos}{ch}% cos hyp
\DeclareMathOperator{\pgcd}{pgcd}% PGCD de deux relatifs
\DeclareMathOperator{\ppcm}{ppcm}% PPCM de deux relatifs
\DeclareMathOperator{\Bary}{Bar}% barycentre
\DeclareMathOperator{\VectEngendre}{Vect}%sous espace endgendre
\DeclareMathOperator{\Opp}{Opp}%sous espace vectoriel des opposés
\DeclareMathOperator{\Inv}{Inv}%sous espace vectoriel des invariants
\DeclareMathOperator{\Dim}{dim}%dimension dun espace vectoriel
\DeclareMathOperator{\rg}{rang}%rang d'une famille de vecteurs
%\DeclareMathOperator{\deg}{deg}%degré d'un polynôm
\DeclareMathOperator{\val}{val}%valuation d'un polynôme
\DeclareMathOperator{\Com}{Com}%comatrice
\DeclareMathOperator{\grad}{\vect{\mathbf{grad}}}%gradient
\DeclareMathOperator{\rot}{\vect{\mathbf{rot}}}%rotationnel
%
\newcommand{\cercle}[2]{\mathcal{C}(#1,#2)}% cercle
\newcommand{\congru}[3]{#1 \equiv #2 \pmod {#3}} %congru
\newcommand{\noncongru}[3]{#1 \not\equiv #2 \pmod {#3}} %non-congru
\newcommand{\expc}[2]{#1 \e^{\ii #2}}% exp complexe
\renewcommand{\bar}{\overline}
\newcommand{\Uv}{\vect{u}}
\newcommand{\V}{\vect{v}}
\newcommand{\W}{\vect{w}}
%% pour les dérivées
\newcommand{\D}{\mathrm{d}}
\newcommand{\derived}[2]{\dfrac{\D #1}{\D #2}}
\newcommand{\deriveds}[2]{\dfrac{\D^2 #1}{\D {#2}^2}}
\newcommand{\derivep}[2]{\dfrac{\dr #1}{\dr #2}}
\newcommand{\deriveps}[2]{\dfrac{\dr^2 #1}{\dr {#2}^2}}
\newcommand{\derivepc}[3]{\dfrac{\dr^2 #1}{\dr {#2} \dr {#3}}}
%%
\newcommand{\dr}{\partial{}}
%%
\newcommand\Hrule{\noindent \rule[0mm]{\linewidth}{0.5pt}}
%
%% Diagramme de Venn
\newcommand{\En}{(-4,-1) rectangle (4,4)}
\newcommand{\An}{(0,0) ++(135:2) circle (2)}
\newcommand{\Bn}{(0,0) ++(45:2) circle (2)}
\newcommand{\AnB}{(0,0) arc (-45:45:2) arc (135:225:2)}
\newcommand{\AuB}{(0,0) arc(-135:135:2) arc(45:315:2)}
\newcommand{\AmB}{(0,0) arc (225:135:2) arc (45:315:2)}
\newcommand{\BmA}{(0,0) arc(-135:135:2) arc(45:-45:2)}


%%pour la profondeur des sections, sous sections, ...

\setcounter{secnumdepth}{3}
\setcounter{tocdepth}{3}

%pour modifier l'apprence de la sectino, subsection,...
% \makeatletter
% \renewcommand{\thesection}{\@Roman\c@section}
% \makeatother
% \makeatletter
% \renewcommand{\thesubsubsection}{\alph{subsubsection}}
% \makeatother

%\addtolength{\oddsidemargin}{-1cm}
%\addtolength{\evensidemargin}{-1cm}

%pour enlever le num de chap dans le titre de la section
%\renewcommand*\thesection{\arabic{section}}
%pour augment le nombre maximal de colonne matrice
\setcounter{MaxMatrixCols}{20}
%pour changer le nom des tableaux
%\renewcommand{\tablename}{Tableau}
\addto\captionsfrench{\def\tablename{\bsc{Tableau}}}
%
%pour changer le carré en CQFD
%\renewcommand{\qedsymbol}{C.Q.F.D.}

%pour rajouter un argument [] pour la largeur des interlignes
% \makeatletter
% \renewcommand*\env@matrix[1][\arraystretch]{%
%   \edef\arraystretch{#1}%
%   \hskip -\arraycolsep
%   \let\@ifnextchar\new@ifnextchar
%   \array{*\c@MaxMatrixCols c}}
% \makeatother
\setlength{\parindent}{0em}
\setlength{\parskip}{1em}